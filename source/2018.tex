\documentclass[11pt, final]{article}
\usepackage{tar}

\begin{document}

    \reporttitle{Trustees Annual Report\\01 Apr 2017 -- 31 Mar 2018}

    \section{Reference and Administration Details}
    \begin{tabular}{l l}
        \textbf{Charity Name:} & The UK Python Association \\
        \textbf{Charity Number:} & 1173471 \\
        \textbf{Principal Address:} & 3 Hillside Close, Helsby, Cheshire WA6 9LB\\
    \end{tabular}

        \subsection{Trustees}
        \begin{tabular}{l l}
            Owen Campbell & Kristian Glass\\
            Peter Inglesby & Kirk Northrop\\
            Chloe Parkes & Daniele Procida\\
            Nicholas Tollervey & \\
        \end{tabular}
        
    \section{Objectives and Activities}
    The primary objective of the UKPA is: `To advance education for the public benefit in the use and understanding of the Python programming language within the UK by organising, presenting and promoting Python related conferences, meetings and events at venues in the UK.'\\
    \\
    The activities of the UKPA are to organise and run the annual PyCon UK conference.

    \section{Structure, Governance and Management}

        \subsection{Governing Document}
        The UKPA's governing document is its constitution, the current version of which is available at https://github.com/PyconUK/ukpa-constitution/releases/latest. 

        \subsection{UKPA Members}
        The UKPA membership is the electoral body to which the trustees are accountable.
        Membership of the UKPA is open to anyone and is routinely offered to those who attend the annual PyCon UK conference. The trustees are working on a formal membership policy, to be presented at the next AGM.

        \subsection{Trustee Selection}
        The Trustees were elected at the AGM at PyCon UK 2017, in accordance with section 13 of the UKPA's constitution.

	    \subsection{Risk and Internal Control}
        The trustees have identified the major risks to which they believe the UKPA is exposed. These are:
        
          \subsubsection{Reduction or Loss of Members}
            If there were a reduction in membership of the UKPA then there would have to be a contraction, consolidation or closure of its activities.

            \subsubsection{Injury to Members, Conference Attendees or Trustees}
            The UKPA requires insurance to cover this potential risk and the trustees intend to put that insurance in place once the UKPA has sufficient funds to cover the cost of the premium.
            
            \subsubsection{Reduction or Loss of Volunteers}
            The UKPA is totally reliant upon volunteers to run and administer the activities of the UKPA. If there were a reduction in the number of volunteers to an unacceptable level, then there would have to be a contraction, consolidation or closure of activities.
            
            \subsubsection{Cancellation of the Conference}
            The financial reserves policy of the UKPA is intended to allow the funding of one year's conference. In the event of cancellation, the reserves would be used to fund the resulting liabilities.
            
            If such a cancellation were to occur whilst the UKPA reserves were below the policy amount, there would have to be a contraction, consolidation or closure of activities.
            
            \subsubsection{Insufficient Conference Ticket Sales}
            The financial reserves policy of the UKPA is intended to allow the funding of one year's conference. In the event of insufficient ticket sales, the reserves would be used to fund the resulting liabilities.
            
            If this were to occur whilst the UKPA reserves were below the policy amount, there would have to be a contraction, consolidation or closure of activities.
              
            \subsubsection{Loss of Sponsors}
            The financial reserves policy of the UKPA is intended to allow the funding of one year's conference. In the event of loss of sponsors, the reserves would be used to fund the resulting liabilities.
            
            If this were to occur whilst the UKPA reserves were below the policy amount, there would have to be a contraction, consolidation or closure of activities.

    \section{Achievements and Performance}
        The UKPA was successfully registered with the Charity Commission in June 2017 and its first AGM was held on Saturday 28\textsuperscript{th} October, 2017 during the PyCon UK conference in Cardiff.
        
        At the AGM, the board of trustees was elected and, the following day, held their initial trustees meeting where the conference director for 2018 was appointed and it was agreed that the financial period for the UKPA would end on March 31\textsuperscript{st} each year. 
        
    \section{Financial Review}

        \subsection{Reserves Policy}
        The UKPA has undertaken a review of its potential liabilities in the face of foreseeable risks and determined that a reasonable sum to hold in reserve is that required to cover the deposits for the annual conference: \pounds37.5K

        At the end of March 2018, the UKPA held reserves of \pounds0.
        
        All funds raised from previous PyCon UK conferences are currently held by PyCon UK Society Ltd. The intention is that these funds will eventually be transferred to the UKPA and the limited company will be closed. However, that transfer has not taken place since the UKPA has does not yet have a bank account. The trustees intend to open an account as soon as possible and to undertake the transfer once the 2018 conference is complete.
        
        At 31-Mar-2018, the UKPA had no bank account, had received no income nor incurred any expenditure and had no assets or liabilities.
        
        \subsection{Investment Policy}
        The UKPA has adopted a low risk strategy and all funds are held in cash using only mainstream banks or building societies.

    \section{Declaration}
    The trustees declare that they have approved the trustees report above.\\
    \\
    Signed on behalf of the charity's trustees:\\
    \\
    \begin{tabular}{l l}
        \textbf{Signature:}\vspace{2cm}\\
        \textbf{Date:} & 21 May 2018\vspace{1cm}\\   
        \textbf{Full Name:} & Owen Campbell \\
    \end{tabular}

\end{document}

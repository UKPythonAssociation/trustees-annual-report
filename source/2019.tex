\documentclass[11pt, final]{article}
\usepackage{tar}

\begin{document}

    \reporttitle{Trustees Annual Report\\01 Apr 2018 -- 31 Mar 2019}

    \section{Reference and Administration Details}
    \begin{tabular}{l l}
        \textbf{Charity Name:} & The UK Python Association \\
        \textbf{Charity Number:} & 1173471 \\
        \textbf{Principal Address:} & c/o Acconomy, Arena Business Centre, Holyrood Close, Poole BH17 7FJ\\
    \end{tabular}

        \subsection{Trustees}
        \begin{tabular}{l l}
            Owen Campbell & Kristian Glass\\
            Tania Allard & Cecilia Liao\\
            Daniele Procida &\\
        \end{tabular}
        
    \section{Objectives and Activities}
    The primary objective of the UKPA is: `To advance education for the public benefit in the use and understanding of the Python programming language within the UK by organising, presenting and promoting Python related conferences, meetings and events at venues in the UK.'\\
    \\
    The activities of the UKPA are to organise and run the annual PyCon UK conference.

    \section{Structure, Governance and Management}

        \subsection{Governing Document}
        The UKPA's governing document is its constitution, the current version of which is available at https://github.com/PyconUK/ukpa-constitution/releases/latest. 

        \subsection{UKPA Members}
        The UKPA membership is the electoral body to which the trustees are accountable.
        Membership of the UKPA is open to anyone and is routinely offered to those who attend the annual PyCon UK conference. The trustees are working on a formal membership policy, to be presented at the next AGM.

        \subsection{Trustee Selection}
        At the AGM in 2018, Kristian Glass resigned as a trustee, and he, Tania Allard and Cecilia Liao were elected as trustees in accordance with section 13 of the UKPA's constitution. At the first trustees meeting, Daniele Procida was appointed Conference Director for the forthcoming year and thus became an ex-officio trustee.

	    \subsection{Risk and Internal Control}
        The trustees have identified the major risks to which they believe the UKPA is exposed. These are:
        
          \subsubsection{Reduction or Loss of Members}
            If there were a reduction in membership of the UKPA then there would have to be a contraction, consolidation or closure of its activities.

            \subsubsection{Injury to Members, Conference Attendees or Trustees}
            The UKPA requires insurance to cover this potential risk and the trustees intend to put that insurance in place once the UKPA has sufficient funds to cover the cost of the premium.
            
            \subsubsection{Reduction or Loss of Volunteers}
            The UKPA is totally reliant upon volunteers to run and administer the activities of the UKPA. If there were a reduction in the number of volunteers to an unacceptable level, then there would have to be a contraction, consolidation or closure of activities.
            
            \subsubsection{Cancellation of the Conference}
            The financial reserves policy of the UKPA is intended to allow the funding of one year's conference. In the event of cancellation, the reserves would be used to fund the resulting liabilities.
            
            If such a cancellation were to occur whilst the UKPA reserves were below the policy amount, there would have to be a contraction, consolidation or closure of activities.
            
            \subsubsection{Insufficient Conference Ticket Sales}
            The financial reserves policy of the UKPA is intended to allow the funding of one year's conference. In the event of insufficient ticket sales, the reserves would be used to fund the resulting liabilities.
            
            If this were to occur whilst the UKPA reserves were below the policy amount, there would have to be a contraction, consolidation or closure of activities.
              
            \subsubsection{Loss of Sponsors}
            The financial reserves policy of the UKPA is intended to allow the funding of one year's conference. In the event of loss of sponsors, the reserves would be used to fund the resulting liabilities.
            
            If this were to occur whilst the UKPA reserves were below the policy amount, there would have to be a contraction, consolidation or closure of activities.

    \section{Achievements and Performance}
        After a lengthy and comprehensive review, the Trustees determined that the formation of a subsidiary trading company would be the best solution for delivering the PyCon UK conference for the future. Lawyers have been engaged and the first draft of that company's articles of association have been prepared. It is anticipated that the company will be registered by the end of 2019.
        
    \section{Financial Review}

        \subsection{Reserves Policy}
        The UKPA has undertaken a review of its potential liabilities in the face of foreseeable risks and determined that a reasonable sum to hold in reserve is that required to cover the deposits for the annual conference: \pounds37.5K

        At the end of March 2019, the UKPA held reserves of \pounds5K.
        
        All funds raised from previous PyCon UK conferences are currently held by PyCon UK Society Ltd (PUKSL). The intention is that these funds will be transferred to the new subsidiary trading company  and PUKSL will be closed.
        
        \subsection{Investment Policy}
        The UKPA has adopted a low risk strategy and all funds are held in cash using only mainstream banks or building societies.

    \section{Declaration}
    The trustees declare that they have approved the trustees report above.\\
    \\
    Signed on behalf of the charity's trustees:\\
    \\
    \begin{tabular}{l l}
        \textbf{Signature:}\vspace{2cm}\\
        \textbf{Date:} & 20 August 2019\vspace{1cm}\\   
        \textbf{Full Name:} & Owen Campbell \\
    \end{tabular}

\end{document}
